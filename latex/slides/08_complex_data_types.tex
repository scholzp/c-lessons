\input{slides_template}	% nothing to do here
\usepackage[english]{babel}

\usepackage[utf8]{inputenc}

\newcommand{\course}{
	C introduction
}

\author{
}

\lstset{
	language = C,
	showspaces = false,
	showtabs = false,
	showstringspaces = false,
	escapechar = @,
	belowskip=-1.5em
}

\def\ContinueLineNumber{\lstset{firstnumber=last}}
\def\StartLineAt#1{\lstset{firstnumber=#1}}
\let\numberLineAt\StartLineAt

\makeatletter
\def\beamer@verbatimreadframe{%
	\begingroup%
	\let\do\beamer@makeinnocent\dospecials%
	\count@=127%
	\@whilenum\count@<255 \do{%
		\advance\count@ by 1%
		\catcode\count@=11%
	}%
	\beamer@makeinnocent\^^L% and whatever other special cases
	\beamer@makeinnocent\^^I% <-- PATCH: allows tabs to be written to temp file
	\endlinechar`\^^M \catcode`\^^M=12%
	\@ifnextchar\bgroup{\afterassignment\beamer@specialprocessframefirstline\let\beamer@temp=}{\beamer@processframefirstline}}%
\makeatother
 % TODO modify this if you have not already done so

% meta-information
\newcommand{\topic}{
	Complex data types
}

% nothing to do here
\title{\topic}
\supertitle{\course}
\date{}

% the actual document
\begin{document}

\maketitle

\begin{frame}{Contents}
	\tableofcontents
\end{frame}

\section{Composite data types}
\subsection{}

\begin{frame}{Limits of primitive data types}
	Primitive data types are fine as long as you want to
	\begin{itemize}
		\item Store a single value that does not depend on other variables
		\item Store a sequence of values of the same type with a constant length \\
		$\rightarrow$ \textit{arrays}
	\end{itemize} \ \\ \ \\
	However, it is not possible to
	\begin{itemize}
		\item Compose variables of different data types to a compound structure \\
		$\rightarrow$ \textit{composite data types}
		\item Have a variable that can only attain certain values \\
		$\rightarrow$ \textit{enumerations}
		\item Have a sequence with an adjustable length \\
		$\rightarrow$ soon...
	\end{itemize}
\end{frame}

\begin{frame}[fragile = singleslide]{Data records}
	Composite data types are derived from primitive data types. You can store any number of primitive variables in one composite variable.
	\begin{itemize}
		\item The composite variable is called \textit{structure} and has the type \textit{struct}
		\item The primitive variables are called \textit{members} of that structure
	\end{itemize} \ \\ \ \\
	Defining a new composite type "\textit{struct} person":
	\begin{lstlisting}[numbers=none]
struct person {		/* struct <identifier> */
	int id;
	int age;		/* block for member declaration */
	char name[32];
};					/* end declaration with ';' */
\end{lstlisting}
	A \textit{struct} variable is at least as large as all of its members.
\end{frame}
\begin{frame}[fragile = singleslide]{\textit{struct} variables}
	Our new type \textit{struct person} can be used to declare variables any where in its scope:
	\begin{lstlisting}[numbers=none]
struct person pers_alice, pers_bob;
\end{lstlisting} \ \\ \ \\
	You can declare a \textit{struct} variable directly in the type definition:
	\begin{lstlisting}[numbers=none]
struct person {
	/* member declaration */
} pers_alice, pers_bob;
\end{lstlisting}
	If we do not need the struct type \textit{person} for further variable declarations, its identifier can be left out.
\end{frame}
\begin{frame}[fragile = singleslide]{Definition and member access}
	To initialize the \textit{struct} members upon declaration, enclose the values in braces as we did it for arrays:
	\begin{lstlisting}[numbers=none]
struct person pers_alice = { 1, 20, "Alice" };
\end{lstlisting} \ \\ \ \\
	To access the struct members, use the struct identifier followed by a '\textbf{.}' and the member identifier:
	\begin{lstlisting}[numbers=none]
printf("%d\n", pers_alice.id);
pers_alice.age++;
\end{lstlisting}
\end{frame}
\begin{frame}[fragile = singleslide]{\textit{struct}s as \textit{struct} members}
	An address is rather complicated:
	\begin{lstlisting}[numbers=none]
struct address {
	int postcode;
	/* ... imagine much more members */
};
\end{lstlisting}
	Now, let the \textit{person} have one:
	\begin{lstlisting}[numbers=none]
struct person {
	struct address contact;
	/* ... and all the other members */
} pers_alice;
\end{lstlisting}
	Access:
\begin{lstlisting}[numbers=none]
pers_alice.contact.postcode = 15430;
\end{lstlisting}
\end{frame}
\begin{frame}[fragile = singleslide]{\textit{union}s}
	\begin{itemize}
		\item Similar to \textit{struct}s, handle them in the same way
		\item However: only one member can be "active"
		\item If you assign a value to a member, all other members become invalid
	\end{itemize} \ \\ \ \\
	Interface between a \textit{list-style} and a \textit{vector-style} implementation:
	\begin{lstlisting}[numbers=none]
union compound {
	int list[3];
	struct {
		int x1, x2, x3;
	} vector;
};
\end{lstlisting}
	The size of a union variable is equal to the size of its largest member.\\
	$\rightarrow$ saving memory
\end{frame}
\section{Enumerations}
\subsection{}
\begin{frame}[fragile = singleslide]{Smart aliases}
	An enumeration consists of identifiers that behave like \textit{constant values}. \\
	It is declared using the keyword \textit{enum}:
	\begin{lstlisting}[numbers=none]
enum light {
	RED,
	YELLOW,
	GREEN
};
\end{lstlisting}
	Now you can assign the values \textit{red}, \textit{yellow} and \textit{green} to variables of the type \textit{enum light}. Internally they are represented as numbers ($red = 0$, $yellow = 1$ etc.), but
	\begin{itemize}
		\item Using the aliases is clear and fancy
		\item No invalid values (like \textit{-1}) can be assigned
	\end{itemize}
\end{frame}
\begin{frame}[fragile = singleslide]{Profit}
	You can determine the values of the constants on your own:
	\begin{lstlisting}[numbers=none]
enum workday {
	MONDAY,			/* 0 */
	TUESDAY,		/* 1 */
	THURSDAY = 3,	/* 3 */
	FRIDAY			/* 4 - implicit (predecessor + 1) */
};
\end{lstlisting}
	However, this can confuse people $\rightarrow$ only use it if there is a good reason. \\ \ \\
	Enumerations provide a nice way to define "global" constants:
	\begin{lstlisting}[numbers=none]
enum { WIDTH = 10, HEIGHT = 20 };
...
char tetris_board[WIDTH][HEIGHT];
\end{lstlisting}
\end{frame}

\section{Style}
\subsection{}
\begin{frame}{Consistency}
	\begin{itemize}
		\item Since complex type definitions heavily rely on blocks, you should use the same coding conventions on them
		\item Let your custom type identifiers start with small letters
	\end{itemize} \ \\ \ \\
	If you define a complex data type, you are very likely going to use it in many different parts of your program. \\
	$\rightarrow$ Have a global type definition, declare the variables in the local context \\ \ \\
	Name \textit{enum} constants in CAPITAL letters to visually seperate them from variables.
\end{frame}
\begin{frame}[fragile = singleslide]{\textbf{typedef}}
	Sometimes you see people writing code like that:
	\begin{lstlisting}[numbers=none]
typedef struct foo {
	/* member declarations */
} bar;
\end{lstlisting}
	This creates the new type \textit{bar} which is nothing more than a \textit{struct foo}. \\ \ \\
	However, this simple fact is hidden for other programmers working on the same project $\rightarrow$ \textbf{possible confusion}.
	\begin{itemize}
		\item Unclear, if \textit{bar} is a composite type at all
		\item If so, is it a \textit{struct} / \textit{union} / \textit{enum} or something really crazy?
	\end{itemize}
\end{frame}
\begin{frame}{Never use \textit{typedef}.}
	\LARGE
	\centering
	Please, avoid using \textit{typedef}.\footnotemark \ \\ \ \\
	\normalsize
	\flushleft
	\begin{uncoverenv}<2->
		Of course, there are situations in which the use of \textit{typedef} makes sense. BUT:
		\begin{itemize}
			\item Not in the C introduction course
			\item Not for simple \textit{struct}s
		\end{itemize}
	\end{uncoverenv}
	\footnotetext[1]{Seriously, never use \textit{typedef}.}
\end{frame}

% nothing to do from here on
\end{document}
