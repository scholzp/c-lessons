\input{../templates/course_definitions}
% This Document contains the information about this course.

% Authors of the slides
\author{
    Lecturers:
	%name1
	Mirko Jantschke,
	%name2
	Pascal Scholz
	\\
	%creaters of the slides
	Created by: Richard Mörbitz, Manuel Thieme
}


% Fancy Logo
\titlegraphic{\hfill\includegraphics[height=1.25cm]{../templates/fsr_logo_cropped}}


\title{C-Lessons}
\subtitle{C standard library - stdlib}
\date{\today}

\usetikzlibrary{tikzmark}


\newcommand{\tikzgraphic}[2]{
\begin{tikzpicture}[scale=#1,font=#2]
	\node (main) [rectangle,draw,inner sep=1em] at (0,0){main.c};
	\node (libc) [rectangle,draw,inner sep=1em] at (5,-3){libc.so};
	\node (stdio) [rectangle,draw,inner sep=1em] at (5,0){stdio.h};
	\node (maino) [rectangle,draw,inner sep=1em] at (0,-3){main.o};
	\node (aout) [rectangle,draw,inner sep=1em] at (2.5,-5){a.out};
	\node at (aout) [above=1.5em] {linking};
	\draw[->] (main) -- (maino) node [left,align=right,midway,text width=5em] {compiling, assembling};
	\draw[->,dotted] (main) -- (stdio) node [above,midway] {includes};
	\draw[->,dotted] (stdio) -- (libc) node [right,midway] {implementation};
	\draw[->] (libc) -- (aout);
	\draw[->] (maino) -- (aout);
\end{tikzpicture}
}

% the actual document
\begin{document}

\maketitle

%-------------------------------------------------------------------------------------------------

\begin{frame}{Contents}
	\tableofcontents
\end{frame}

%-------------------------------------------------------------------------------------------------
\section{Do not reinvent the wheel}
%-------------------------------------------------------------------------------------------------

\begin{frame}[fragile]{The Hitchhiker's Guide to the standard library}
	Many of the problems you will be working on have already been solved.\\
	\bigskip
	These solution are provided in \textit{libraries}. Prefer them over your own code!\\
	\begin{itemize}
		\item they are safer
		\item they are more efficient
		\item you do not have to do everything by yourself
	\end{itemize}
	\bigskip
	We have already used the \textbf{C standard library} for i/o:
\begin{lstlisting}[numbers=none]
#include <stdio.h>	/* Anyone remember this? */
\end{lstlisting}
	This statement includes the \textit{header} file \textbf{stdio.h}.\\
	After compilation, \textit{gcc} will link your program with the C library.\\
\end{frame}

%-------------------------------------------------------------------------------------------------

\begin{frame}{Linking happens}
	\centering
	\tikzgraphic{.9}{\footnotesize}
\end{frame}

%-------------------------------------------------------------------------------------------------
\section{Useful headers}
%-------------------------------------------------------------------------------------------------

\begin{frame}[fragile]{assert.h}
	 \begin{itemize}
	 	\item Contains the \textit{assert()} macro, witch evaluates the truth value of an expression
	 	\item If it is true, nothing happens
	 	\item Else the program aborts and an error message is printed
	 \end{itemize} \ \\ \ \\
	 $\rightarrow$ useful to avoid undefined behaviour / worse errors at runtime \\ \ \\
	 We can also use it if we just want to test things:
	 \begin{lstlisting}[numbers=none]
unsigned int input;
printf("Enter a one-digit decimal number:\n");
scanf("%d", &input);
assert(input < 10);
\end{lstlisting}
\end{frame}

%-------------------------------------------------------------------------------------------------

\begin{frame}[fragile]{math.h}
	\begin{itemize}
		\item Declares a lot of mathematical functions
		\item Finally you are able to calculate square roots, logarithms, etc.
		\item Most of those functions have \textit{double} arguments and return values
	\end{itemize} \ \\ \ \\
	If you use functions from \textit{math.h}, add the \textit{-lm} as the \textbf{last} option to \textit{gcc} to avoid errors:
	\begin{lstlisting}[numbers=none]
gcc main.c -lm
\end{lstlisting}
\end{frame}

%-------------------------------------------------------------------------------------------------

\begin{frame}[fragile]{stdio.h}
	\begin{itemize}
		\item Declares the basic functions to read and write data
		\item You know \textit{printf()} and \textit{scanf()}, but there is more:
		\item Characters, unprocessed and formatted strings
		\item Command line I/O and file access
		\item Many functions for high-level file management
	\end{itemize} \ \\ \ \\
	As an example, \textit{puts()} can be used instead of \textit{printf()} if you have a basic string without placeholders - '\textit{\textbackslash n}' is added automatically:
	\begin{lstlisting}[numbers=none]
puts("Hello World!");
/* Equivalent to printf("Hello World!\n") */
\end{lstlisting}
\end{frame}

%-------------------------------------------------------------------------------------------------

\begin{frame}{stdlib.h}
	This probably is the most powerful header providing various different functionalities. Here is just an excerpt:
	\begin{itemize}
		\item \textit{EXIT\_SUCCESS} and \textit{EXIT\_FAILURE} constants as an alternative to returning \textit{0} or \textit{1} at the end of \textit{main()}
		\item Alternative ways to exit the program
		\item Generation of pseudo-random numbers
		\item Search and sorting function
		\item Dynamic memory management
	\end{itemize}
		\dots and more things you have not even heard of (yet)
\end{frame}

%-------------------------------------------------------------------------------------------------

\begin{frame}{string.h}
	\begin{uncoverenv}<2->
		Wait! Strings? \\ \ \\
	\end{uncoverenv}
	\begin{uncoverenv}<3->
		Yes, there are strings in C. \\
		They are just handled differently from what you would expect. \\ \ \\
		\textit{string.h} is crucial if you want to work with C strings seriously. \\
		We will use some of the functions declared there in later lessons.
	\end{uncoverenv}
\end{frame}

%-------------------------------------------------------------------------------------------------

\begin{frame}{time.h}
	\begin{itemize}
		\item Data types to store different time formats
		\item Functions to get the calendar and cpu time
		\item Functions to format time values
		\item Functions to measure and calculate time differences
	\end{itemize} \ \\ \ \\
	Handling time usually is quite complicated, but with the help of \textit{time.h} it gets a lot easier.\\
	\ \\
	Measure the execution time of your programs to see how efficient they are!
	Be carefull as TICKS
\end{frame}

%-------------------------------------------------------------------------------------------------
\section{Man page}
%-------------------------------------------------------------------------------------------------

\begin{frame}[fragile]{Documentation}
	Learning all the library functions is way less effective than knowing where to look them up quickly. \\
	\textit{Man page} is a Unix tool containing documentation of programs, system calls and libraries - such as the C standard library. \\ \ \\
	To access a certain man page, just type:
	\begin{lstlisting}[numbers=none, basicstyle=\itshape\small]
$ man page
\end{lstlisting} \ \\ \ \\
Example for \textit{printf()}:
	\begin{lstlisting}[numbers=none]
$ man printf
\end{lstlisting}
However, this describes the shell command \textit{printf}.
\end{frame}

%-------------------------------------------------------------------------------------------------

\begin{frame}[fragile]{Effective use of \textit{man}}
	Man has many sections, library functions are in \#3. \\
	Write the section number between \textit{man} and the page:
	\begin{lstlisting}[numbers=none]
$ man 3 printf
\end{lstlisting} \ \\ \ \\
	To get all pages \textit{printf} occurs in, use the \textit{-k} option:
	\begin{lstlisting}[numbers=none]
$ man -k printf
\end{lstlisting} \ \\ \ \\
	If you need more information on \textit{man} - it has its own man page:
	\begin{lstlisting}[numbers=none]
$ man man
\end{lstlisting}
\end{frame}

%-------------------------------------------------------------------------------------------------
\section{Related Tasks}
%-------------------------------------------------------------------------------------------------

\begin{frame}{Alea iacta est - Using rand()}
    \href{http://fsr.github.io/c-lessons/exercises/14_alea_iacta_est.html}{Task as online:} \newline
    To solve the following tasks, you need to use some functions that may be new to you.
    No explanation is given on them, use man to find out how they work.

    Write a program that simulates a dice using the rand() function.
    \newline 
    \newline
    Experts: Simulate “real” randomness that cannot be recreated.
\end{frame}
%-------------------------------------------------------------------------------------------------

\begin{frame}{Know your architecture - Using sizeof()}
    \href{http://fsr.github.io/c-lessons/exercises/15_know_your_architecture.html}{Task as online:} \newline
    Write a program, that prints the size of every (or at least as much until you get bored) primitive datatype (in bytes).
    Primitive datatypes in C are:
    \begin{itemize}
     \item char
     \item int
     \item float
     \item double
    \end{itemize}

    Extra: Try to combine this types with long (for example long int).
\end{frame}
%-------------------------------------------------------------------------------------------------

\begin{frame}{Runtime of arithmetics - Using time.h}
    Not a Task which is online, but it's \href{http://fsr.github.io/c-lessons/exercises/14_alea_iacta_est.html}{solution}: \newline
    Write a Programm which takes the time for each basic arithmetic operation (Addition, Subtraction, Multiplikation, Division) and Modulo. Use \textit{int} 
    as type so the calculated time is comparable. 
    \newline
    \newline
    Hint: You might need to measure more than one calculation for each operation as modern processers are really fast.
    \newline 
    \newline
    Experts: Compare the runtime for division with and without using the multiplcative inverse (calculate 100/m and 100*1/m for example). 
\end{frame}
%-------------------------------------------------------------------------------------------------
\end{document}
