\input{../templates/course_definitions}
% This Document contains the information about this course.

% Authors of the slides
\author{
    Lecturers:
	%name1
	Mirko Jantschke,
	%name2
	Pascal Scholz
	\\
	%creaters of the slides
	Created by: Richard Mörbitz, Manuel Thieme
}


% Fancy Logo
\titlegraphic{\hfill\includegraphics[height=1.25cm]{../templates/fsr_logo_cropped}}


\title{C-Lessons}
\subtitle{Variables}
\date{\today}

\usetikzlibrary{tikzmark}
\begin{document}

\begin{frame}
	\titlepage
\end{frame}
\begin{frame}{Overview}
	\setbeamertemplate{section in toc}[sections numbered]
	\tableofcontents
\end{frame}

%-------------------------------------------------------------------------------------------------
\section{Using functions}
\subsection{}

%-------------------------------------------------------------------------------------------------
\begin{frame}[fragile]{\only<1>{Remember the main \textit{function}?}%
                       \only<2->{Defining functions}}
	\begin{columns}[T]
		\column{.33\textwidth}<3->
		data type of the returned value or \textit{void}, if nothing is returned
		\column{.33\textwidth}<4->
		unique name to refer the function, same rules as for variable identifiers
		\column{.34\textwidth}<5->
		argument declarations, seperated by commas or \textit{void}, if there are none
	\end{columns}
	\begin{textblock}{7}(1.5,5.5)
        \begin{tikzpicture}[x=1.2cm,y=0.9cm]
            \only<3>{\draw[red] (-.3,1) edge[out=270,in=90,->,shorten >=0ex] (0,-0.5);}
            \uncover<4->{\draw (-.3,1) edge[out=270,in=90,->,shorten >=0ex] (0,-0.5);}
            \only<4>{\draw[red] (3.5,1) edge[out=270,in=90,->,shorten >=0ex] (2.25,-0.5);}
            \uncover<5->{\draw (3.5,1) edge[out=270,in=90,->,shorten >=0ex] (2.25,-0.5);}
            \only<5>{\draw[red] (7,1) edge[out=270,in=90,->,shorten >=.0ex] (4.5,-0.5);}
            \uncover<6->{\draw (7,1) edge[out=270,in=90,->,shorten >=.0ex] (4.5,-0.5);}
        \end{tikzpicture}
	\end{textblock}
	\ \\\ 
	
    \begin{onlyenv}<1>
	\begin{lstlisting}[numbers=none]
int main(void) {
	/* code happens */
	return 0;
}
\end{lstlisting}
    \end{onlyenv}
    \begin{onlyenv}<2->
	\begin{lstlisting}[numbers=none,basicstyle=\itshape\small]
return_type identifier(argument_list) {
	function_body
	return expression;
}
\end{lstlisting}
    \end{onlyenv}
	\	\\\	\\\ \\
	\begin{columns}[T]
		\column{.5\textwidth}<6->
		just as in \textit{main()}, all statements are put in here
		\column{.5\textwidth}<7->
		value this function returns or empty, if the return value is \textit{void}
	\end{columns}
	\begin{textblock}{7}(.5,8.5)
		\begin{tikzpicture}
			\only<6>{\draw[red] (1.8,0) edge[out=90,in=180,->,shorten >=0ex] (0,2);}
			\uncover<7->{\draw (1.5,0) edge[out=90,in=180,->,shorten >=0ex] (0,2);}
			\only<7>{\draw[red] (8,0) edge[out=90,in=270,->,shorten >=0ex] (2.4,1.5);}
			\uncover<8->{\draw (8,0) edge[out=90,in=270,->,shorten >=0ex] (2.4,1.5);}
		\end{tikzpicture}
	\end{textblock}
\end{frame}


%-------------------------------------------------------------------------------------------------
\begin{frame}[fragile]{Passing arguments}
	\begin{itemize}
		\item Each value is assigned to the parameter at the same position in the argument list (and therefore must have the same type)
	\end{itemize}		
	\begin{lstlisting}[basicstyle=\scriptsize]
#include <stdio.h>	
	
void shift_character(char character, unsigned offset) {
	printf("%c\n", (character + offset) % 255);
}

int random_number(void) {
	return 4;	// chosen by fair dice roll.
				// guaranteed to be random.
}

int main(void) {
	int offset = 10;
	shift_character('c', offset);
	printf("%d\n", random_number());
	return 0;
}
\end{lstlisting}
\end{frame}

%-------------------------------------------------------------------------------------------------
\section{More on scopes}
\subsection{}
%-------------------------------------------------------------------------------------------------
\begin{frame}[fragile]{Global variables}
	\begin{itemize}
		\item Variables defined outside any function
		\item Scope: from line of declaration to end of program
	\end{itemize}
	\begin{lstlisting}
int globe = 42;

void foo(void) {
	globe = 23;
}

int main(void) {
	printf("%d\n", globe);	/* Prints 42 */
	foo();
	printf("%d\n", globe);	/* Prints 23 */
	...
\end{lstlisting}
	Altering them in one function may have \textbf{side effects} on other functions $\rightarrow$ use them rarely.
\end{frame}
\begin{frame}[fragile]{Where not to call functions}
	Since a function's scope starts at the line of its definition, having two functions \textit{f()} and \textit{g()} calling each other is not possible:
	\begin{lstlisting}
void f(int i) {
	...
	g(42);	/* What is g? */
}

void g(int i) {
	...
	f(42);
}
\end{lstlisting}
	In that case, \textit{g()} is called outside its scope. Changing the order does not work either.
\end{frame}

%-------------------------------------------------------------------------------------------------
\begin{frame}[fragile]{Prototypes}
	Like variables, functions can also be \textit{declared}:
	\begin{lstlisting}[numbers=none,basicstyle=\itshape\footnotesize]
return_type identifier(argument list);
\end{lstlisting}
	\begin{itemize}
		\item It's similar to a definition, just replace the function body by a \textit{;}
		\item Declared functions must also be defined any where in the program
		\item In the argument list, only types matter $\rightarrow$ identifiers \textbf{can} be left out
	\end{itemize}
	\begin{lstlisting}
void g(int i);	/* better do not leave the identifier out */

void f(int i) {
	...
	g(42);		/* Now a call of g() can be compiled */
}

void g(int i) {...}	/* g() definition, similar to f() */
\end{lstlisting}
\end{frame}

%-------------------------------------------------------------------------------------------------
\begin{frame}[fragile]{Better program structure}
	To avoid problems like that above, it is a common practise to \textit{declare} all functions at the top of the file and define them below the main function:
	\begin{lstlisting}
void f(int i);
void g(int i);

int main(void) {
	...
}

void f(int i) {
	...
	g(42);
}

/* g() definition, similar to f() */
\end{lstlisting}
\end{frame}

%-------------------------------------------------------------------------------------------------
\begin{frame}[fragile]{Good documentation style}
    Add a documentation comment to each function prototype:
    \begin{lstlisting}
/*
 * Get the sum of two numbers.
 * num:    input number
 */
int factorial(int num);
\end{lstlisting}
    There are frameworks such as \emph{doxygen} that parse your comments and
    create a fancy HTML documentation:
    \begin{lstlisting}[escapeinside={}]
/**
 * @brief Get the sum of two numbers.
 * @param num1  first number
 * @param num2  second number
 * @return      sum of num1 and num2
 */
int add(int num1, int num2);
\end{lstlisting}
\end{frame}

%-------------------------------------------------------------------------------------------------
\begin{frame}{Functions in functions}
	You \textbf{could} define functions in functions.\footnotemark
	
	\footnotetext[1]{Just saying.}
\end{frame}

%-------------------------------------------------------------------------------------------------
\section{Recursion}
\subsection{}
%-------------------------------------------------------------------------------------------------
\begin{frame}[fragile]{Recursive functions}
	\begin{itemize}
		\item Functions calling themselves
		\item Used to implement many mathematical algorithms
		\item Easy to think up, but they run slow
	\end{itemize} \ \\ \ \\
	Careful:
	\begin{lstlisting}
void foo(void) {
	foo();
}
\end{lstlisting}
	creates an infinite loop.\footnote{And, at some point, a program crash (\textit{stack overflow})} \\
	There must always be an \textit{exit condition} if using recursion!
\end{frame}

%-------------------------------------------------------------------------------------------------
\begin{frame}[fragile]{Exponentiation}
As an example, take a look at this function calculating $base^{exponent}$:
	\begin{lstlisting}
int power(int base, int exponent) {
	if (exponent == 0)
		return 1;
	return base * power(base, exponent - 1);
}
\end{lstlisting}
	\begin{itemize}
		\item $a^{0} = 1 \rightarrow$ \textit{power(a, 0}) just returns \textit{1}
		\item $a^{b} = a \cdot a^{b-1} \rightarrow$ recursive call of \textit{power(a, b-1)}
	\end{itemize}
\end{frame}
%----------------------------------------------------------------------------------------------------

\begin{frame}[fragile]{Example {power(2,3)}}
	\begin{lstlisting}
int power(int base, int exponent) {
	if (exponent == 0)
		return 1;
	return base * power(base, exponent - 1);
}\end{lstlisting}
	\begin{columns}[T]
		\column{.5\textwidth}<1->
		1st call: power(2,3) \tikzmark{c11}
		\column{.5\textwidth}<8->
		\tikzmark{c12} return \begin{math}{2 \cdot {power(2,2)} = 2 \cdot 4 = 8} \end{math}  
	\end{columns}
    \vspace{5mm}
    \begin{columns}[T]
		\column{.5\textwidth}<2->
		2nd call: power(2,2) \tikzmark{c21}
		\column{.5\textwidth}<7-> 
		\tikzmark{c22} return \begin{math}{2 \cdot {power(2,1)} = 2 \cdot 2 = 4} \end{math} 
	\end{columns}
    \vspace{5mm}
    \begin{columns}[T]
		\column{.5\textwidth}<3->
		3rd call: power(2,1)   \tikzmark{c31}
		\column{.5\textwidth}<6->
		\tikzmark{c32} return \begin{math}{2 \cdot \textit{power(2,0)} = 2 \cdot 1 = 2} \end{math}   
	\end{columns}
    \vspace{5mm}
	\begin{columns}[T]
		\column{.5\textwidth}<4->
		4th call: power(2,0)   \tikzmark{c41}
		\column{.5\textwidth}<5->
		\tikzmark{c42} return \begin{math}{1} \end{math} 
	\end{columns}

	\begin{textblock}{7}(1.5,4.5)
        \begin{tikzpicture}[overlay, remember picture, yshift=.25\baselineskip, shorten >=.5pt, shorten <=.5pt]
            \uncover<2->{\draw [->] ({pic cs:c11}) [bend left] to ({pic cs:c21});}
            \uncover<3->{\draw [->] ({pic cs:c21}) [bend left] to ({pic cs:c31});}
            
            \uncover<4->{\draw [->] ({pic cs:c31}) [bend left] to ({pic cs:c41});}
            \uncover<5->{\draw [->] ({pic cs:c41}) [bend right] to ({pic cs:c42});}
            \uncover<6->{\draw [->] ({pic cs:c42}) [bend left] to ({pic cs:c32});}
            \uncover<7->{\draw [->] ({pic cs:c32}) [bend left] to ({pic cs:c22});}
            \uncover<8->{\draw [->] ({pic cs:c22}) [bend left] to ({pic cs:c12});}
        \end{tikzpicture}
	\end{textblock}
\end{frame}
%----------------------------------------------------------------------------------------------------

\section{Related Task}
\begin{frame}{Power and faculty}
    \href{http://fsr.github.io/c-lessons/exercises/11_practising_recursion.htmll}{Task as online:} \newline
    Write a function that takes a numbers a from the user and calculates \begin{math}{a!}\end{math}
    \newline 
    \newline
    Experts: Write a program that calculates the fibonacci number of a fib(a) of the user input a.
\end{frame}


\end{document}
