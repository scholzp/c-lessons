\input{../templates/course_definitions}
% This Document contains the information about this course.

% Authors of the slides
\author{
    Lecturers:
	%name1
	Mirko Jantschke,
	%name2
	Pascal Scholz
	\\
	%creaters of the slides
	Created by: Richard Mörbitz, Manuel Thieme
}


% Fancy Logo
\titlegraphic{\hfill\includegraphics[height=1.25cm]{../templates/fsr_logo_cropped}}


\title{C-Lessons}
\subtitle{Complex data types}
\date{\today}

\usetikzlibrary{tikzmark}

% the actual document
\begin{document}

\maketitle

\begin{frame}{Contents}
	\tableofcontents
\end{frame}

%-------------------------------------------------------------------------------------------------
\section{Composite data types}
%-------------------------------------------------------------------------------------------------

\begin{frame}{Limits of primitive data types}
	Primitive data types are fine as long as you want to
	\begin{itemize}
		\item Store a single value that does not depend on other variables
		\item Store a sequence of values of the same type with a constant length \\
		$\rightarrow$ \textit{arrays}
	\end{itemize} \ \\ \ \\
	However, it is not possible to
	\begin{itemize}
		\item Compose variables of different data types to a compound structure \\
		$\rightarrow$ \textit{composite data types}
		\item Have a variable that can only attain certain values \\
		$\rightarrow$ \textit{enumerations}
		\item Have a sequence with an adjustable length \\
		$\rightarrow$ soon...
	\end{itemize}
\end{frame}

%-------------------------------------------------------------------------------------------------

\begin{frame}[fragile]{Data records}
	Composite data types are derived from primitive data types. You can store any number of primitive variables in one composite variable.
	\begin{itemize}
		\item The composite variable is called \textit{structure} and has the type \textit{struct}
		\item The primitive variables are called \textit{members} of that structure
	\end{itemize} \ \\ \ \\
	Defining a new composite type "\textit{struct} person":
	\begin{lstlisting}[numbers=none]
struct person {		/* struct <identifier> */
	int id;
	int age;		/* block for member declaration */
	char name[32];
};					/* end declaration with ';' */
\end{lstlisting}
	A \textit{struct} variable is at least as large as all of its members.
\end{frame}

%-------------------------------------------------------------------------------------------------

\begin{frame}[fragile]{\textit{struct} variables}
	Our new type \textit{struct person} can be used to declare variables any where in its scope:
	\begin{lstlisting}[numbers=none]
struct person pers_alice, pers_bob;
\end{lstlisting} \ \\ \ \\
	You can declare a \textit{struct} variable directly in the type definition:
	\begin{lstlisting}[numbers=none]
struct person {
	/* member declaration */
} pers_alice, pers_bob;
\end{lstlisting}
	If we do not need the struct type \textit{person} for further variable declarations, its identifier can be left out.
\end{frame}

%-------------------------------------------------------------------------------------------------

\begin{frame}[fragile]{Definition and member access}
	To initialize the \textit{struct} members upon declaration, enclose the values in braces as we did it for arrays:
	\begin{lstlisting}[numbers=none]
struct person pers_alice = { 1, 20, "Alice" };
\end{lstlisting} \ \\ \ \\
	To access the struct members, use the struct identifier followed by a '\textbf{.}' and the member identifier:
	\begin{lstlisting}[numbers=none]
printf("%d\n", pers_alice.id);
pers_alice.age++;
\end{lstlisting}
\end{frame}

%-------------------------------------------------------------------------------------------------

\begin{frame}[fragile]{\textit{struct}s as \textit{struct} members}
	An address is rather complicated:
	\begin{lstlisting}[numbers=none]
struct address {
	int postcode;
	/* ... imagine much more members */
};
\end{lstlisting}
	Now, let the \textit{person} have one:
	\begin{lstlisting}[numbers=none]
struct person {
	struct address contact;
	/* ... and all the other members */
} pers_alice;
\end{lstlisting}
	Access:
\begin{lstlisting}[numbers=none]
pers_alice.contact.postcode = 15430;
\end{lstlisting}
\end{frame}

%-------------------------------------------------------------------------------------------------

\begin{frame}[fragile]{\textit{union}s}
	\begin{itemize}
		\item Size of the union is equal to the size of the largest member
		\item All member variables share the same memory
		\item Everytime you access a member, you access the same memory
		\item If you assign a value to a member, all other members become invalid (as they share the same memory)
	\end{itemize} \ \\ 
	Let's assume the follwing union:
	\begin{lstlisting}[numbers=none]
union compound {
	int a;      /* size = 32 bit */
	float b;    /* size = 32 bit */
	char c[8];  /* size = 64 bit */
};
\end{lstlisting}
 $\rightarrow$ This union will be 64bit large.
\end{frame}

%-------------------------------------------------------------------------------------------------
\begin{frame}[fragile]{\textit{union}s}
Now let's see what happens, if we use the union:
	\begin{lstlisting}[numbers=left]
/* cool stuff */
    union c u;
	u.a = 0;
	u.b = 0;
	printf("C: %s\n", u.c);	
	char str[8] = {'a','b','c','d','e','f','g','\0'}; 
	strcpy(u.c, str); /* Function from string.h */
	printf("A: %d\n", u.a);
	printf("B: %f\n", u.b);	
	printf("C: %s\n", u.c);	
/* more cool stuff */ \end{lstlisting}
And the Output will be:
	\begin{lstlisting}[numbers=left]
    C:                      // prints '\0'
	A: 1684234849bvgvf
    B: 16777999408082104352768.000000
    C: abcdefg\end{lstlisting}
\end{frame}

%-------------------------------------------------------------------------------------------------
\section{Enumerations}
%-------------------------------------------------------------------------------------------------

\begin{frame}[fragile]{Smart aliases}
	An enumeration consists of identifiers that behave like \textit{constant values}. \\
	It is declared using the keyword \textit{enum}:
	\begin{lstlisting}[numbers=none]
enum light {
	RED,
	YELLOW,
	GREEN
};    \end{lstlisting}
	Now you can assign the values \textit{red}, \textit{yellow} and \textit{green} to variables of the type \textit{enum light}. Internally they are represented as numbers ($red = 0$, $yellow = 1$ etc.), but
	\begin{itemize}
		\item Using the aliases is clear and fancy
		\item No invalid values (like \textit{-1}) can be assigned
	\end{itemize}
\end{frame}

%-------------------------------------------------------------------------------------------------

\begin{frame}[fragile]{Profit}
	You can determine the values of the constants on your own:
	\begin{lstlisting}[numbers=none]
enum workday {
	MONDAY,			/* 0 */
	TUESDAY,		/* 1 */
	THURSDAY = 3,	/* 3 */
	FRIDAY			/* 4 - implicit (predecessor + 1) */
};\end{lstlisting}
	However, this can confuse people $\rightarrow$ only use it if there is a good reason. \\ \ \\
	Enumerations provide a nice way to define "global" constants:
	\begin{lstlisting}[numbers=none]
enum { WIDTH = 10, HEIGHT = 20 };
...
char tetris_board[WIDTH][HEIGHT];
\end{lstlisting}
\end{frame}

%-------------------------------------------------------------------------------------------------
\section{Style}
%-------------------------------------------------------------------------------------------------
\begin{frame}{Consistency}
	\begin{itemize}
		\item Since complex type definitions heavily rely on blocks, you should use the same coding conventions on them
		\item Let your custom type identifiers start with small letters
	\end{itemize} \ \\ \ \\
	If you define a complex data type, you are very likely going to use it in many different parts of your program. \\
	$\rightarrow$ Have a global type definition, declare the variables in the local context \\ \ \\
	Name \textit{enum} constants in CAPITAL letters to visually seperate them from variables.
\end{frame}

%-------------------------------------------------------------------------------------------------

\begin{frame}[fragile]{\textbf{typedef}}
	Sometimes you see people writing code like that:
	\begin{lstlisting}[numbers=none]
typedef struct foo {
	/* member declarations */
} bar;
\end{lstlisting}
	This creates the new type \textit{bar} which is nothing more than a \textit{struct foo}. \\ \ \\
	However, this simple fact is hidden for other programmers working on the same project $\rightarrow$ \textbf{possible confusion}.
	\begin{itemize}
		\item Unclear, if \textit{bar} is a composite type at all
		\item If so, is it a \textit{struct} / \textit{union} / \textit{enum} or something really crazy?
	\end{itemize}
\end{frame}

%-------------------------------------------------------------------------------------------------
\section{Related Task}
%-------------------------------------------------------------------------------------------------
\begin{frame}{Cristmas Task}
    \href{https://github.com/scholzp/c-lessons/tree/master/solutions/Santas_sledge}{Solution to this task}
    \newline
    Christmas Task:
    Write a program, that calculates, if the reindeer of santas sledge have enough pullforce to pull the weight of the presents in his sledge.
    We will need 3 structs (for example data see next slide):
    \begin{itemize}
     \item struct reindeer, that contains age, gender and pullforce as members
     \item struct present, that contains weight and price as members
     \item struct sledge, that contains max. 8 reindeer, max. 20 presents and a maxTransportWeight as members
    \end{itemize}
    For the gender of the reindeer use an enum with m or w.\newline\newline
    If the reindeer don't have the pullforce to pull the loaded sledge, the programm should printe how many kilosgrams of pullforce are left. So santa knows how many and which of his reindeers he has to add.

\end{frame}

%-------------------------------------------------------------------------------------------------
\begin{frame}{Cristmas Task 2}
    Examples data for struct presents:
    \begin{itemize}
     \item Computer: weight = 7kg, price = 850€
     \item Bookshelf: weight = 55kg, price = 519€
     \item Fridge: weight = 150kg, price = 750€
    \end{itemize}
    Expample data for reindeer:
    \begin{itemize}
     \item rudolph: age = 12, gender = m, pullforce = 50kg
    \end{itemize}
    Bonus: Santa also has to pay a 2 percent fee for his insurance of the worth of his cargo. Calculate the total insurance he has to pay!
\end{frame}
%-------------------------------------------------------------------------------------------------


\begin{frame}{Circles}
    \href{http://fsr.github.io/c-lessons/exercises/21_Circles.html}{Task as online}
    \newline
    A point in 2D consists of two positions: x and y (both float). A circle consists of a 
    centre (point), a radius, a circumference and an area (all float).
    \newline
    Write a program that reads two positions and a radius from the command line. They should
    be stored in a struct circle as described above along with its remaining parameters 
    (calculate them!). Then, print the resulting circle.
    \newline 
    \newline
    Experts: Write a function that takes two circles as arguments and returns a circle that 
    goes through their centres.

\end{frame}
%-------------------------------------------------------------------------------------------------

\end{document}
