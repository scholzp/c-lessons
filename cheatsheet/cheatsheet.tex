\documentclass[a4paper,10pt]{article}

\usepackage[utf8]{inputenc}
\usepackage[table]{xcolor}
%for defining new colors 
\usepackage{color, colortbl}
%listings for Code
\usepackage{listings}
\usepackage{array}
\usepackage{tabularx}

%to use lineending in cells
\usepackage{makecell}
\usepackage{multirow}
\usepackage{typearea}
\usepackage{graphicx}
\usepackage{float}
\usepackage{hyperref}
\usepackage[a4paper,left=2cm,right=2cm,top=2.5cm,bottom=2cm,bindingoffset=5mm]{geometry}
\lstset{language=C, morekeywords={printf}, keywordstyle=\color{red}}
\definecolor{LightYellow}{rgb}{1,1,0.85}

\title{C Cheatsheet}
\author{}
\date{}


\pdfinfo{%
  /Title    (C Cheatsheet)
  /Author   ()
  /Creator  ()
  /Producer ()
  /Subject  ()
  /Keywords ()
}

\begin{document}
{\LARGE C-Cheatsheet}

\section{Printf Formatstring}

\begin{lstlisting}
printf(%[flags][width][.percision][length]type);
\end{lstlisting} 

%%%%%%%%%%%%%%%%%%%%%%%%%%%%%%%%%%%%%%%%%%%%%%%%%%%%%%%%%%%
%flags
\begin{center}
    \rowcolors{1}{white}{LightYellow}
    \begin{tabularx}{\textwidth}{|l|X|}
        \hline
        \multicolumn{2}{|c|}{Mögliche Flags} \\
        \hline
            Flag   & Beschreibung\\
        \hline
            -               & Linksausrichten der Ausgabe innerhalb des Platzhalters\\
            +               & Stellt positiven Zahlen ein Vorzeichen voran.\\
            (Leerzeichen)   & Stellt positiven Zahlen ein Leerzeichen voran\\
            0               & Platzhalter wird mit Nullen aufgefüllt (anstatt Leerzeichen)\\
            \#              & Für G und g Typen: Nachfolgende Nullen werden nicht
                            entfernt.
                            Für F, f, e, E, g, G Typen: Output enthält immer einen '.' für Nachkommstellen.
                            Für o, x, X Typen: Die Texte 0, 0x, 0X.\\ 
        \hline
    \end{tabularx}
\end{center}
%%%%%%%%%%%%%%%%%%%%%%%%%%%%%%%%%%%%%%%%%%%%%%%%%%%%%%%%%%%
%width field
\begin{center}
    \rowcolors{1}{white}{LightYellow}
    \begin{tabularx}{\textwidth}{|l|X|}
        \hline
        \multicolumn{2}{|c|}{Width Feld} \\
        \hline
            Zeichen   & Beschreibung\\
        \hline
            n (Gannzahl)    & Definiert, dass der ausgeben Parameter (in
                            Kombination  mit d, f,..) mindestens die Länge n hat. Sollte er  weniger Zeichen haben, wird nach links mit Füllzeichen (Standard Leezeichen, mit Flag 0 mit Nullen) aufgefüllt.\\ 
            \verb|*|d       & Siehe Zeile drüber. Einziger Unterschied, dass der
                            Wert nun in einer Variable steht.\\ 
        \hline
    \end{tabularx}
\end{center}
%%%%%%%%%%%%%%%%%%%%%%%%%%%%%%%%%%%%%%%%%%%%%%%%%%%%%%%%%%5
%percision field
\begin{center}
    \rowcolors{1}{white}{LightYellow}
    \begin{tabularx}{\textwidth}{|l|X|}

        \hline
        \multicolumn{2}{|c|}{Precision Feld} \\
        \hline
            Zeichen   & Beschreibung\\
        \hline
            n (Gannzahl)    &   Definiert, dass der ausgeben Parameter (in
                            Kombination mit d, f,..) maximal die Länge n hat. Sollte er mehr Zeichen haben, werden diese abgeschnitten (Standard Leezeichen, mit Flag 0 mit Nullen) aufgefüllt.\\ 
            \verb|*|d      &  Siehe Zeile drüber. Einziger Unterschied, dass der
                            Wert nun in einer Variable steht.\\ 
        \hline
    \end{tabularx}
\end{center}
%%%%%%%%%%%%%%%%%%%%%%%%%%%%%%%%%%%%%%%%%%%%%%%%%%%%%%%%%%%
%length field
Das Lenght-Feld wird benutz um die Größe des Datentypes zu übermitteln. Wird es frei gelassen, wird dieses aus dem Typen ermittelt.
\begin{center}
    \rowcolors{1}{white}{LightYellow}    
    \begin{tabularx}{\textwidth}{|l|X|}
        \hline
        \multicolumn{2}{|c|}{Length Feld} \\
        \hline
            Zeichen   & Beschreibung\\
        \hline
            hh             & Bewirkt, dass printf ein Gannzahl-Argument  erwartet,
                            welches so groß ist wie ein int aber von einem char bereitsgestellt wird.\\
            h              & Bewirkt, dass printf ein Gannzahl-Argument erwartet,
                            welches so groß ist wie ein int aber von einem short bereitsgestellt wird.\\
            l              & Bewirkt, dass printf ein Gannzahl-Argument von der Größe eines    
                            long erwartet\\
            ll             & Bewirkt, dass printf ein Gannzahl-Argument von der Größe eines    
                            long long erwartet\\
            L              & Bewirkt, dass printf ein Fließkomma-Argument erwartet,
                            welches so groß ist wie ein long double\\
            z              & Bewirkt, dass printf ein Ganzzahl-Argument erwartet,
                            welches so groß ist wie ein size\_t.\\
            j              & Bewirkt, dass printf ein Ganzzahl-Argument erwartet,
                            welches so groß ist wie ein intmax\_t.\\
            t              & Bewirkt, dass printf ein Ganzzahl-Argument erwartet,
                            welches so groß ist wie ein ptrdiff\_t.\\
        \hline
    \end{tabularx}
\end{center}

\begin{figure}[H]
  \caption{Printf Beispiel.}
  \centering
    \includegraphics[width=.8\textwidth]{printf}
    \\
  \href{https://commons.wikimedia.org/wiki/File:Printf.svg}{Quelle: Wikipedia, abgerufen am 02.11.2018}
\end{figure}
%%%%%%%%%%%%%%%%%%%%%%%%%%%%%%%%%%%%%%%%%%%%%%%%%%%%%%%%%%%
\begin{center}
    \rowcolors{1}{white}{LightYellow}
    \begin{tabularx}{\textwidth}{|l|X|}
        \hline
        \multicolumn{2}{|c|}{Typ Feld (Format Specifier)} \\
        \hline
            Zeichen   & Beschreibung\\
        \hline
            \%       & Gibt einfach ein \% aus.\\
            d, i           & Zusammen mit einem übergebenen Wert vom Typ \textit{int} wird eine vorzeichenbehaftete
                             Dezimalzahl ausgegeben.\\
            u              & Gibt vorzeichenlose Dezimalzahl aus.\\
            f, F           & Stellt Werte vom Typ \textit{doulbe} als Festkommezahl dar.\\
            e, E           & Für \textit{doulbe} Werte. Stellt diese im wissenschaftlichen Format mit Exponent dar.
                             e oder E unterscheiden nur den Buchstaben, der für den Exponent verwendet wird. Der Exponent hat immer mindestens 2 Zeichen, unter Windows 3.\\
            g, G           & Für Werte vom typ \textit{double}. Wählt je nach Größe entweder die Festkomma- oder die
                            wissenschaftliche (mit Exponent) Darstellung \\
            x, X           & Gibt einen \textit{unsigned int} als Hexadezimalzahl aus.\\
            o              & Gibt einen \textit{unsigned int} als Octalzahl aus.\\
            s              & Für einen Null-terminieten String.\\  
            c              & Gibt ein Zeichen vom Typ \textit{char} aus.\\ 
            p              & Gibt die Adresse eine Pointers aus.\\
            a, A           & Gibt einen Wert vom Typ \textit{double} im als Hexadezimalzahl aus.\\
            n              & Keine Ausgabe. Schreibt Anzahl erfolgreich ausgegebener Zeichen in einen Pointer des
                             Typs \textit{int}\\
        \hline
    \end{tabularx}
\end{center}
%%%%%%%%%%%%%%%%%%%%%%%%%%%%%%%%%%%%%%%%%%%%%%%%%%%%%%%%%%%
\newcolumntype{L}[1]{>{\raggedright\let\newline\\\arraybackslash\hspace{0pt}}m{#1}}
\newcolumntype{C}[1]{>{\centering\let\newline\\\arraybackslash\hspace{0pt}}m{#1}}
\newcolumntype{R}[1]{>{\raggedleft\let\newline\\\arraybackslash\hspace{0pt}}m{#1}}
    
\section{Basisdatentypen in C}
\begin{center}
    \rowcolors{1}{white}{LightYellow}
    \begin{tabularx}{\textwidth}{|L{3.5cm}|L{1.3cm}|X|}
        \hline
        \multicolumn{3}{|c|}{Datentypen in C} \\
        \hline
            Typ  & Specifier & Beschreibung \\
        \hline
            char               &\%c       & Kleinster Datentyp. Enthält eine Ganzzahl, die \textit{signed} oder
                                            \textit{unsigned} sein kann.\\ 
            singed char        &\%c       & Gleiche Größe wie \textit{char}, aber vorzeichenbehaftet. Wertebereich
                                            ist min. -127 bis 127.\\ 
            unsinged char      &\%c       & Gleiche Größe wie \textit{char}, aber vorzeichenlos. Wertebereich
                                            ist min. 0 bis 255.\\
            \mbox{short}, \mbox{short int}, \mbox{signed short}, \mbox{signed short int}
                               &\%hi      & Ganzzahldatentypen mit Wertebereich von -32.767 bis +32.767.\\
            \mbox{unsigned short}, \mbox{unsigned short int}, \mbox{unsigned}
                               &\%hu      & Ganzzahldatentypen mit Wertebereich von 0 bis +65.535.\\
            \mbox{int}, \mbox{signed}, \mbox{signed int}
                       &\%i oder \%d      & Basisdatentyp für vorzeichenbehaftete Ganzahlen.\\
            \mbox{long}, \mbox{long int}, \mbox{signed long}, \mbox{signed long int}
                               &\%li      & Vorzeichenbehafteter Ganzzahldatentypen mit Wertebereich von min. -2.147.483.647 bis +2.147.483.647.\\
            \mbox{unsigned long}, \mbox{unsigned long int}
                               &\%lu      & Ganzzahldatentypen mit Wertebereich von min. 0 bis +4.294.967.295.\\
            \mbox{long long}, \mbox{long long int}, \mbox{signed long long}, \mbox{signed long long int}
                               &\%li      & Vorzeichenbehafteter Ganzzahldatentypen mit Wertebereich von min. -9.223.372.036.854.775.807 bis +9.223.372.036.854.775.807.\\
            \mbox{unsigned long long}, \mbox{unsigned long long int}
                               &\%lu      & Ganzzahldatentypen mit Wertebereich von min. 0 bis
                               +18.446.744.073.709.551.615.\\
            float               &\%f       & Typ für Gleitkommazahlen, meistens einfach Genauigkeit
                                                                nach IEEE 754 (32 Bit)\\ 
            double              &\%lf       & Typ für Gleitkommazahlen, meistens doppelte Genauigkeit
                                                                nach IEEE 754 (64 Bit)\\
            long double         &\%Lf       & Typ für Gleitkommazahlen, mit erweiterter Genauigkeit. Größe ist       
                                            implementierungsabhängig, aber min. so groß wie \textit{double}.\\
        \hline
    \end{tabularx}
\end{center}

\end{document}
