\documentclass[a4paper,10pt]{scrartcl}

\usepackage[utf8]{inputenc}
\usepackage[table]{xcolor}
%for defining new colors 
\usepackage{color, colortbl}
%listings for Code
\usepackage{listings}
%to use lineending in cells
\usepackage{makecell}

\lstset{language=C,morekeywords={printf}, keywordstyle=\color{red}}
\definecolor{LightYellow}{rgb}{1,1,0.7}

\title{C Cheatsheet}
\author{}
\date{}


\pdfinfo{%
  /Title    (C Cheatsheet)
  /Author   ()
  /Creator  ()
  /Producer ()
  /Subject  ()
  /Keywords ()
}

\begin{document}
\maketitle
\begin{lstlisting}
printf(%[flags][width][.percision][length]type);
\end{lstlisting} 

%flags
\begin{center}
    \rowcolors{1}{white}{LightYellow}
        \begin{tabular}{|c|l|}
        \hline
        \multicolumn{2}{|c|}{Possible Flags} \\
        \hline
            Flag   & Beschreibung\\
        \hline
            -               & Linksausrichten der Ausgabe innerhalb des Platzhalters\\
            +               & Stellt positiven Zahlen ein Vorzeichen voran.\\
            (Leerzeichen)   & Stellt positiven Zahlen ein Leerzeichen voran\\
            0               & Platzhalter wird mit Nullen aufgefüllt (anstatt Leerzeichen)\\
            \verb|#|        & \makecell[l]{ Für G und g Typen: Nachfolgende Nullen werden nicht
                            entfernt.\\
                            Für F, f, e, E, g, G Typen: Output enthält immer einen '.' für Nachkommstellen.\\
                            Für o, x, X Typen: Die Texte 0, 0x, 0X.}\\ 
        \hline
    \end{tabular}
\end{center}

%width field
\begin{center}
    \rowcolors{1}{white}{LightYellow}
        \begin{tabular}{|c|l|}
        \hline
        \multicolumn{2}{|c|}{Width Feld} \\
        \hline
            Zeichen   & Beschreibung\\
        \hline
            n (Gannzahl)    &  \makecell[l]{ Definiert, dass der ausgeben Parameter (in
                            Kombination mit d, f,..) mindestens \\die Länge n hat. Sollte er  weniger Zeichen haben, wird nach links mit Füllzeichen \\(Standard Leezeichen, mit Flag 0 mit Nullen) aufgefüllt.}\\ 
            \verb|*|d       &  \makecell[l]{Siehe Zeile drüber. Einziger Unterschied, dass der
                            Wert nun in einer Variable steht.}\\ 
        \hline
    \end{tabular}
\end{center}

%percision field
\begin{center}
    \rowcolors{1}{white}{LightYellow}
        \begin{tabular}{|c|l|}
        \hline
        \multicolumn{2}{|c|}{Precision Feld} \\
        \hline
            Zeichen   & Beschreibung\\
        \hline
            n (Gannzahl)    &  \makecell[l]{ Definiert, dass der ausgeben Parameter (in
                            Kombination mit d, f,..) maximal \\die Länge n hat. Sollte er mehr Zeichen haben, werden diese abgeschnitten \\(Standard Leezeichen, mit Flag 0 mit Nullen) aufgefüllt.}\\ 
            \verb|*|d      &  \makecell[l]{Siehe Zeile drüber. Einziger Unterschied, dass der
                            Wert nun in einer Variable steht.}\\ 
        \hline
    \end{tabular}
\end{center}

%length field
\begin{center}
    \rowcolors{1}{white}{LightYellow}
        \begin{tabular}{|c|l|}
        \hline
        \multicolumn{2}{|c|}{Length Feld} \\
        \hline
            Zeichen   & Beschreibung\\
        \hline
            hh             & Bewirkt, dass printf ein Gannzahl-Argument  erwartet, welches so groß ist wie ein int aber von einem char bereitsgestellt wird.\\
            h              & Bewirkt, dass printf ein Gannzahl-Argument erwartet, welches so groß ist wie ein int aber von einem short bereitsgestellt wird.\\
            %TODO
            l              & Bewirkt, dass printf ein Gannzahl-Argument erwartet, welches so groß ist wie ein int aber von einem short bereitsgestellt wird.\\
            \verb|*|d      &  \makecell[l]{Siehe Zeile drüber. Einziger Unterschied, dass der
                            Wert nun in einer Variable steht.}\\ 
        \hline
    \end{tabular}
\end{center}
Das Lenght-Feld wird benutz um die Größe des Datentypes zu übermitteln. Wird es frei gelassen, wird dieses aus dem Typen ermittelt
    
\section{Primitive Datentypen}

\end{document}
